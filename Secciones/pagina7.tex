\section{VENTAJAS Y DESVENTAJAS} 

\begin{enumerate}[7.]

	\begin{center}
	\item VENTAJAS
	\end{center}
  \\ Se ahorra el papel. 
  \\ Se disminuye la  necesidad de espacios en una biblioteca.  
  \\ Se puede acceder a información desde cualquier parte del mundo.  
  \\ La información se puede imprimir, grabar, mandar por correo electrónico. 
  \\ Mayor acceso a informacion. 
  \\ Se puede organizar mejor el acervo  y se puede acumular mas  informacion que en una biblioteca común.
  \\
	\begin{center}
	\item DESVENTAJAS
	\end{center}
	\\ Se ahorra el papel. 
  \\ Disminuyo la comunidad lectora en bibliotecas. 
  \\ Hace mas falta de interés lectora. 
  \\ Disminuye la falta de pensamiento en diversos ámbitos.  
  \\ La mitad de las personas no cuentan con acceso a Internet. 
  \\ Gasto en mantenimiento de los equipos. 
	\\
    \item JUSTIFICACION
\\ La Biblioteca Virtual  pretende ser un espacio vivo y dinámico para el trabajo, la búsqueda de información, el dialogo y el intercambio. Donde la información se renueva y enriquezca constantemente. Para poder conseguir los objetivos planteados hemos diferenciado varias partes en la Biblioteca Virtual de Tecnología Educativa.

\end{enumerate} 
