\section{PROBLEMA} 

\begin{enumerate}[1.]
    
    Las bibliotecas siempre han estado al tanto de nuevas tecnologías que permitan desempeñar de una manera mejor su papel de brindar información. Además las infraestructuras informáticas y de telecomunicaciones que actualmente poseen diversas instituciones tanto públicas como privadas, permiten pensar en llevar a otro nivel el uso de estas.
    Actualmente se utliza de manera manual haciéndolo en un determinado tiempo de espera prolongado afectando al estudiante, el encargado presenta estos problemas al momento de la búsqueda de un libro, por autor, editorial y fecha de publicación por ser manejable, no dan rapidez ni solución concreta, cabe rescatar que no poseen un registro de cada libro que se pueda llevar un estudiante de la institución, ni una organización de cada asignatura de manera automatizada, a la llegada de un nuevo libro no contiene anotaciones para poder implementar dicho libro a la biblioteca , no contener un sistema automatizado acapara mucha información ya que se desperdicia.
La motivación principal de este proyecto es por lo tanto mejorar el servicio de préstamo de libros y que produzca en una mayor agilidad y eficacia tanto para el usuario como el empleado de estos servicios bibliotecarios..


\end{enumerate} 
