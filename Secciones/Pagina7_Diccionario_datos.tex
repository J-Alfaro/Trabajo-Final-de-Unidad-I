\documentclass{article}
\usepackage[utf8]{inputenc}

\begin{document}

\section{DESCRIPCIÓN DICCIONARIO DE DATOS}

\subsection{Tabla Nro. 1: Diccionario de Datos Libro:}

\begin{tabular}{|p{25mm}|l|p{20mm}|p{50mm}|}\hline
 
VARIABLE  & TIPO DE DATOS  & CARACTER & DESCRIPCIÓN VARIABLE\\\hline
AutorLibro & String & 50 & Es propio el nombre del autor del libro\\\hline
EstadoDisponible & bool & 25 & Estado disponible de cada libro en la biblioteca.\\\hline
IdLibro & Int & 10 & Código de cada libro agregado en la tabla Libro.\\\hline
TituloLibro & String & 10 & Título del Libro propio.\\\hline
\end{tabular}
\\
\\
\subsection{Tabla Nro. 2: Diccionario de Datos Reserva:}

\begin{tabular}{|p{25mm}|l|p{20mm}|p{50mm}|}\hline
 
VARIABLE  & TIPO DE DATOS  & CARACTER & DESCRIPCIÓN VARIABLE\\\hline
EstadoReserva & String & 10 & Estado de la Reserva.\\\hline
FechaReserva & DateTime & 50 & Fecha en la cual esta ocupada la reserva del libro (Recojo).\\\hline
IdLibro & Int & 10 & Código de cada libro agregado en la tabla Libro.\\\hline
IdReserva & int & 10 & Código de Usuario de la tabla Usuario\\\hline
\end{tabular}
\\
\\
\subsection{Tabla Nro. 3: Diccionario de Datos Préstamo:}

\begin{tabular}{|p{25mm}|l|p{20mm}|p{50mm}|}\hline
 
VARIABLE  & TIPO DE DATOS  & CARACTER & DESCRIPCIÓN VARIABLE\\\hline
IdPrestamo & Int & 11 & ID de cada libro es propio.\\\hline
EstadoPrestamo & String & 10 & Estado de la reserva.\\\hline
FechaReserva & DateTime  & 10 & Contraseña con la cual el usuario se autentica en el sistema\\\hline
IdLibro & Int & 10 & Código de cada libro agregado en la tabla Libro.\\\hline
IdReserva & int & 10 & Código de cada Reserva de la tabla Reserva.\\\hline
IdUsuario & Int & 10 & código de Usuario de la tabla Usuario\\\hline
\end{tabular}
\\
\\
\subsection{Tabla Nro. 4: Diccionario de Datos Usuario:}

\begin{tabular}{|p{25mm}|l|p{20mm}|p{50mm}|}\hline
 
VARIABLE  & TIPO DE DATOS  & CARACTER & DESCRIPCIÓN VARIABLE\\\hline
CodigoUsuario & String & 8 & Código del Usuario asignado y propio. \\\hline
IdUsuario & Int & 10 & código de Usuario de la tabla Usuario\\\hline
NombreUsuario & String & 50 & Nombre de cada Usuario agregado en la tabla Usuario.\\\hline
\end{tabular}



\end{document}
