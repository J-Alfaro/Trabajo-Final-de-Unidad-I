\section{MARCO TEORICO} 

\begin{enumerate}[1.]
	
	La lectura es super importante, y la biblioteca a cumplido una especial labor en la enseñanza y el aprendizaje, pues esta ejerce una influencia sobre el logro académico en los estudiantes, las bibliotecas sirven para propiciar, favorecer y estimular la lectura, pues ofrece tanto a profesores como estudiantes recursos digitales, electrónicos e impresos. 
	
	Las bibliotecas deben ofrecer mucho más que libros, es un lugar de reunión para los estudiantes, en el cual se pueden explorar y debatir ideas, en el manifiesto de la UNESCO en 1994 sobre las bibliotecas públicas, reconocen que la participación constructiva y la consolidación de la democracia dependen tanto de una educación agradable como de un acceso libre y sin límites al conocimiento, el pensamiento, la cultura y la información, por lo cual la biblioteca escolar contribuye a formar una sociedad más democrática mediante el acceso equitativo al conocimiento y a la información.
	
	Tienen una gran influencia en los resultados del aprendizaje, pues el desarrollo de las habilidades lectoras complementadas con el uso de los recursos pueden dar mejores resultados a la hora de investigar un tema o realizar una exposición, puesto que son complementarias tanto a las explicaciones del profesor como de búsqueda de temas de interés o de temas no comprendidos. 
	
El objetivo de este proyecto es crear una biblioteca digital de nueva generación.
Es decir, donde se puedan almacenar, indexar y buscar tanto el contenido de un documento como sus metadatos asociados.
Por ello es necesario entender cada uno de los conceptos que componen la aplicación por separado para comprender la importancia 	del producto final del 	proyecto, así como el conocimiento que se requiere para lograr el alcance final.
Los conceptos a tratar son los siguientes: Bibliotecas, la cual será la	herramienta de desarrollo a utilizar para implementar el 	motor de búsqueda
requerido. .
    
\end{enumerate} 
